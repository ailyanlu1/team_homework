\documentclass[UTF8]{ctexart}
\title{More Queries to Array...题解}
\author{唐适之}
\date{}
\newcommand{\myparagraph}[1]{\paragraph{#1}\mbox{}\\}
\usepackage[top=1in, bottom=1in, left=1.25in, right=1.25in]{geometry}
\usepackage{enumitem}

\begin{document}
\maketitle

\myparagraph{算法1}

赋值操作O(n)扫描赋值,查询操作O(nk)扫描查询。

时间复杂度O(mnk),期望得分10分。

\myparagraph{算法2}

在算法1的基础上预处理i的0至5次方(1≤i≤n),使查询操作复杂度降低到O(n)。这30\%的数据k较大、询问较多较长使算法一不能通过。

时间复杂度O(mn),期望得分30分。

\myparagraph{算法3、4}

我们可以对0≤j≤5维护$i^j a_i$的区间和(1≤i≤n)。考虑一次询问l,r,k,例如k=2,询问的和为$\sum_{i=l}^r  a_i \cdot (i-(l-1))^2 = \sum_{i=1}^r  a_i \cdot i^2 - \sum_{i=1}^r  a_i \cdot 2i(l-1)^2 + \sum_{i=1}^r  a_i \cdot (l-1)^2$,便转换为分别询问0、1、2次$i^j a_i$的区间和。一般地,展开后用i整理可以将询问转换为0至k次的区间和,以此可以实现在知道$i^j a_i$的区间和的情况下O(k)查询。预处理$i^j$(1≤i≤n,0≤j≤5)的前缀和,即可实现插入中O(k)算出一个区间0至5次幂的区间和。以上是在假设维护了区间和的情况下,现在使用数据结构维护。

算法3:使用分块维护区间和,时间复杂度为$O(m\sqrt{n}k)$,空间复杂度O(n),按常数不同,期望得分50至100分。

算法4:使用线段树维护区间和,时间复杂度$O(mk\log_2{n})$,空间复杂度O(n),期望得分100分,这是最优做法。

\myparagraph{算法5}

如果没有预处理$i^j$(1≤i≤n,0≤j≤5)的前缀和,那么插入操作就稍微麻烦一点。如果k≤2,1次和2次情况可以分别使用等差数列和公式和平方和公式O(1)算出一个区间的区间和。

这部分期望得分10分。

\myparagraph{算法6}

由算法5拓展,可以推出0至5次方和的公式。例如立方和公式可以按下面的方法推出:
$$(r+1)^4-r^4=4r^3+6r^2+4r+1$$
$$r^4-(r-1)^4=4(r-1)^3+6(r-1)^2+4(r-1)+1$$
$$\dots$$
$$(l+1)^4-l^4=4l^3+6l^2+4l+1$$

各式相加得
$$S_3=\frac{1}{4} ((r+1)^4-l^4-6S_2-4S_1-S_0)$$

一般地,可用更低次的区间和表示出较高次的区间和($S_k$表示k次和),即
$$S_k=\frac{1}{k+1} ((r+1)^{k+1}-l^{k+1}-\sum_{p=0}^{k-1} C_{k+1}^p S_p)$$

这样求一段区间各次的区间和就要$O(k^2)$的时间,使用线段树的话插入$O(mk^2\log_2{n})$,查询$O(mk\log_2{n})$,总时间$O(mk^2\log_2{n})$,空间复杂度$O(n)$。这样的时间可以被接受,期望得分100分。

\end{document}
